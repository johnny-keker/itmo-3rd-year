\documentclass[12pt, a4paper]{article}
\usepackage[a4paper, includeheadfoot, mag=1000, left=2cm, right=1.5cm, top=1.5cm, bottom=1.5cm, headsep=0.8cm, footskip=0.8cm]{geometry}
% Fonts
\usepackage{fontspec, unicode-math}
\setmainfont[Ligatures=TeX]{CMU Serif}
\setmonofont{CMU Typewriter Text}
\usepackage[english, russian]{babel}
% Indent first paragraph
\usepackage{indentfirst}
\setlength{\parskip}{5pt}
% Diagrams
\usepackage{graphicx}
\usepackage{float}
% Page headings
\usepackage{fancyhdr}
\pagestyle{fancy}
\renewcommand{\headrulewidth}{0pt}
\setlength{\headheight}{16pt}
%\newfontfamily\namefont[Scale=1.2]{Gloria Hallelujah}
\fancyhead{}
\usepackage{ragged2e}

\usepackage{multirow}

\usepackage{listings}
\lstdefinestyle{lablisting}{
  basicstyle=\scriptsize\ttfamily,
  numbers=left,
  stepnumber=1,
  otherkeywords={EOF, O\_RDONLY, STDIN\_FILENO, STDOUT\_FILENO, STDERR\_FILENO},
  numbersep=10pt,
  showspaces=false,
  showstringspaces=false
}

\graphicspath{ {images/} }

\newcommand{\specialcell}[2][l]{%
  \begin{tabular}[#1]{@{}c@{}}#2\end{tabular}}

\newcommand{\figc}[4]{
  \begin{figure}[h]
  \begin{center}
    \includegraphics[scale=#4]{#1}
    \caption{#2}
    \label{fig:#3}
  \end{center}
  \end{figure}
}

\begin{document}

% Title page
\begin{titlepage}
\begin{center}

\textsc{Национальный исследовательский университет ИТМО\\[4mm]
Факультет программной инженерии и компьютерной техники}
\vfill
\textbf{Учебно-исследовательская работа №3\\[4mm]
по дисципение Сети ЭВМ и телекоммуникации\\[4mm]
Анализ трафика компьютерных сетей утилитой Wireshark\\[16mm]
}
\begin{flushright}
Студент: Саржевский Иван
\\[2mm]Группа: P3302
\end{flushright}
\vfill
г. Санкт-Петербург\\[2mm]
2020 г.

\end{center}
\end{titlepage}

\tableofcontents
\newpage

\justify

\section{Цель}

Цель работы – изучить структуру протокольных блоков данных, анализируя реальный
трафик на компьютере студента с помощью бесплатно распространяемой утилиты
\texttt{Wireshark}.

\section{Утилита \texttt{ping}}

Для анализа трафика, создаваемого утилитой \texttt{ping} был выбран сайт
\textbf{\texttt{www.ias.ru}}.

\figc{ping_headers}{Заголовки протоколов для команды \texttt{ping}.}{p_h}{3.0}

На рисунке \ref{fig:p_h} изображены заголовки различных протоколов, используемых
при передаче запроса.

\subsection{Фрейм}

\figc{ping_frame}{Информация о фрейме команды \texttt{ping}.}{p_f}{3.0}

Структура, представленная на рисунке \ref{fig:p_f}, описывает метаданные
Wireshark для этого запроса - его порядковый номер среди всех записанных,
время прибытия, размер, протокол и цвет выделения в интерфейсе.

\subsection{Ethernet II}

Ethernet II - протокол канального уровня, т.е. описывает передачу данных в
рамках локальной сети. Типичная структура кадра Ethernet II представлена в
таблице \ref{Tab:ethernet_frame}.

\begin{table}[h]
\begin{center}
\begin{small}
  \caption{Структура кадра Ethernet II.}
  \begin{tabular}{| c | c | c | c | c |}
    \hline
    \multicolumn{5}{|c|}{\specialcell[c]{Кадр Ethernet II\\ (от 64-х до 1528-ти байт)}}\\
    \hline
    \multicolumn{3}{|c|}{\specialcell[c]{MAC-заголовок\\ (14 байт)}} & \specialcell[c]{Данные\\ (от 46-ти до 1500 байт)} & --\\
    \hline
    \specialcell[c]{MAC получателя\\ (6 байт)} & \specialcell[c]{MAC отправителя\\ (6 байт)} & \specialcell[c]{Тип протокола\\ (2 байта)} & Данные & \specialcell[c]{CRC\\ (4 байта)}\\
    \hline
  \end{tabular}
  \label{Tab:ethernet_frame}
\end{small}
\end{center}
\end{table}

В данном случае получателем выступает роутер, а отправителем - рабочая машина,
их MAC-адреса записаны в кадр, тип протокола - \texttt{IPv4}, что можно увидеть
на рисунке \ref{fig:p_e}.

\figc{ping_ethernet}{Кадр Ethernet II для \texttt{ping}.}{p_e}{3.0}

\subsection{IPv4}

IPv4 - протокол сетевого уровня. Подробные сведения полях, которые включены в заголовок
протокола, приведены на рисунке \ref{fig:ipv4_h}. Туда включены IP-адреса
отправителя и получателя, длинна заголовка и сообщения, флаги указывающие на
наличие фрагментации данных, промежуточности данного пакета и т. д.

\figc{ipv4_header}{Структура заголовка \texttt{IPv4}.}{ipv4_h}{1.4}

Данные, переданные с использованием протокола \texttt{IPv4} для команды
\texttt{ping} можно увидеть на рисунке \ref{fig:p_ip}.

\newpage

\figc{ping_ipv4}{Данные пакета \texttt{IPv4} для команды \texttt{ping}.}{p_ip}{3.0}

\subsection{Internet Control Message Protocol (ICMP)}

\figc{icmp_header}{Структура заголовка \texttt{ICMP}.}{icmp_h}{1.4}

Данный протокол сетевого уровня используется для передачи служебных сообщений -
кода ошибки в случае исключительной ситуации, кода запрашиваемой операции и
кода подтверждения в случае удачной передачи. Подробная структура заголовка
\texttt{ICMP} приведена на рисунке \ref{fig:icmp_h}.

\figc{ping_icmp}{Данные \texttt{ICMP} для команды \texttt{ping}.}{ping_icmp}{3.0}

Для команды \texttt{ping} структура \texttt{ICMP} представлена на рисунке
\ref{fig:ping_icmp}.

Структура ответов имеет схожую структуру, отличаться они будут типом \texttt{ICMP},
сменой адресов получателя и отправителя, \texttt{timestamp}'ами.

\subsection{Ответы на вопросы}

\begin{enumerate}
  \item Имеет ли место фрагментация исходного пакета, какое поле на это указывает?\\
    -- Да. Но только в том случае, если размер пакета превышает \texttt{Maximum
    Transmission Unit} (\texttt{MTU}), равный для протокола \texttt{Ethernet II}
    1500 байт. Информация о наличии фрагментации содержится во флаге в заголовке
    \texttt{IPv4}.
  \item Какая информация указывает, является ли фрагмент пакета последним или
    промежуточным?\\
    -- Флаг \texttt{More Fragments} в заголовке \texttt{IPv4}.
  \item Чему равно количество фрагментов при передаче ping-пакетов?\\
    -- \texttt{MTU} равен 1500 байт, пакет включает в себя \texttt{IPv4}-заголовок
    (20 байт), \texttt{ICMP}-заголовок (8 байт), и, непосредственно, данные.
    Это означает, что количество фрагментов равно $\lceil ( s + 20 + 8 ) / 1500 \rceil$,
    где $s$ - аргумент \texttt{-s} команды \texttt{ping}. Зависимость количества
    фрагментов от размера пакета приведена в таблице \ref{Tab:frag} и в
    графическом виде на рисунке \ref{fig:frag}.

    \figc{frag}{Зависимость количества фрагментов от размера пакета}{frag}{0.8}

    \begin{table}[h]
    \begin{center}
    \begin{small}
      \caption{Количество фрагментов при разных размерах пакета}
      \begin{tabular}{| c | c | c | c | c | c | c | c | c |}
        \hline
        Размер пакета & 100 &	500 & 1000 & 1500 & 2000 & 3000 & 5000 & 10000\\
        \hline
        Кол-во фраг. & 1 & 1 & 1 & 2 & 2 & 3 & 4 & 7\\
        \hline
      \end{tabular}
      \label{Tab:frag}
    \end{small}
    \end{center}
    \end{table}
  \item Как изменить поле TTL с помощью утилиты ping?\\
    \begin{itemize}
      \item \texttt{Linux} : \texttt{ping -t ttl\_value}
      \item \texttt{Windows} : \texttt{ping -i ttl\_value}
    \end{itemize}
  \item Что содержится в поле данных \texttt{ping}-пакета?\\
    -- В поле данных содержится текущий timestamp, а затем циклически
    повторяющиеся биты от \texttt{00} до \texttt{FF}.\\ \\
    \texttt{0030   00 00 5d 12 04 00 00 00 00 00 10 11 12 13 14 15}\\
    \texttt{0040   16 17 18 19 1a 1b 1c 1d 1e 1f 20 21 22 23 24 25}\\
    \texttt{0050   26 27 28 29 2a 2b 2c 2d 2e 2f 30 31 32 33 34 35}\\
    \texttt{0060   36 37 38 39 3a 3b 3c 3d 3e 3f 40 41 42 43 44 45}\\
    \texttt{                    .       .       .                 }\\
    \texttt{0110   e6 e7 e8 e9 ea eb ec ed ee ef f0 f1 f2 f3 f4 f5}\\
    \texttt{0120   f6 f7 f8 f9 fa fb fc fd fe \textbf{FF 00} 01 02 03 04 05}\\
\end{enumerate}


\end{document}
