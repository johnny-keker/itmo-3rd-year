\documentclass[12pt, a4paper]{article}
\usepackage[a4paper, includeheadfoot, mag=1000, left=2cm, right=1.5cm, top=1.5cm, bottom=1.5cm, headsep=0.8cm, footskip=0.8cm]{geometry}
% Fonts
\usepackage{fontspec, unicode-math}
\setmainfont[Ligatures=TeX]{CMU Serif}
\setmonofont{CMU Typewriter Text}
\usepackage[english, russian]{babel}
% Indent first paragraph
\usepackage{indentfirst}
\setlength{\parskip}{5pt}
% Diagrams
\usepackage{graphicx}
\usepackage{float}
% Page headings
\usepackage{fancyhdr}
\pagestyle{fancy}
\renewcommand{\headrulewidth}{0pt}
\setlength{\headheight}{16pt}
%\newfontfamily\namefont[Scale=1.2]{Gloria Hallelujah}
\fancyhead{}
\usepackage{ragged2e}

\usepackage{multirow}

\usepackage{listings}
\lstdefinestyle{lablisting}{
  basicstyle=\scriptsize\ttfamily,
  numbers=left,
  stepnumber=1,
  otherkeywords={EOF, O\_RDONLY, STDIN\_FILENO, STDOUT\_FILENO, STDERR\_FILENO},
  numbersep=10pt,
  showspaces=false,
  showstringspaces=false
}

\newcommand{\specialcell}[2][l]{%
  \begin{tabular}[#1]{@{}c@{}}#2\end{tabular}}

\begin{document}

% Title page
\begin{titlepage}
\begin{center}

\textsc{Национальный исследовательский университет ИТМО\\[4mm]
Факультет программной инженерии и компьютерной техники}
\vfill
\textbf{Учебно-исследовательская работа №3\\[4mm]
по дисципение Сети ЭВМ и телекоммуникации\\[4mm]
Анализ трафика компьютерных сетей утилитой Wireshark\\[16mm]
}
\begin{flushright}
Студент: Саржевский Иван
\\[2mm]Группа: P3302
\end{flushright}
\vfill
г. Санкт-Петербург\\[2mm]
2020 г.

\end{center}
\end{titlepage}

\tableofcontents
\newpage

\justify

\section{Цель}

Цель работы – изучить структуру протокольных блоков данных, анализируя реальный
трафик на компьютере студента с помощью бесплатно распространяемой утилиты
\texttt{Wireshark}.

\section{Утилита \texttt{ping}}

\end{document}
