\documentclass[12pt, a4paper]{article}
\usepackage[a4paper, includeheadfoot, mag=1000, left=2cm, right=1.5cm, top=1.5cm, bottom=1.5cm, headsep=0.8cm, footskip=0.8cm]{geometry}
% Fonts
\usepackage{fontspec, unicode-math}
\setmainfont[Ligatures=TeX]{CMU Serif}
\setmonofont{CMU Typewriter Text}
\usepackage[english, russian]{babel}
% Indent first paragraph
\usepackage{indentfirst}
\setlength{\parskip}{5pt}
% Diagrams
\usepackage{graphicx}
\usepackage{float}
% Page headings
\usepackage{fancyhdr}
\pagestyle{fancy}
\renewcommand{\headrulewidth}{0pt}
\setlength{\headheight}{16pt}
%\newfontfamily\namefont[Scale=1.2]{Gloria Hallelujah}
\fancyhead{}
\usepackage{ragged2e}

\usepackage{multirow}

\usepackage{listings}
\lstdefinestyle{lablisting}{
  basicstyle=\scriptsize\ttfamily,
  numbers=left,
  stepnumber=1,
  otherkeywords={EOF, O\_RDONLY, STDIN\_FILENO, STDOUT\_FILENO, STDERR\_FILENO},
  numbersep=10pt,
  showspaces=false,
  showstringspaces=false
}

\graphicspath{ {images/} }

\newcommand{\specialcell}[2][l]{%
  \begin{tabular}[#1]{@{}c@{}}#2\end{tabular}}

\newcommand{\figc}[4]{
  \begin{figure}[H]
  \begin{center}
    \includegraphics[scale=#4]{#1}
    \caption{#2}
    \label{fig:#3}
  \end{center}
  \end{figure}
}

\begin{document}

% Title page
\begin{titlepage}
\begin{center}

\textsc{Национальный исследовательский университет ИТМО\\[4mm]
Факультет программной инженерии и компьютерной техники}
\vfill
\textbf{Учебно-исследовательская работа №3\\[4mm]
по дисципение Сети ЭВМ и телекоммуникации\\[4mm]
Анализ трафика компьютерных сетей утилитой Wireshark\\[16mm]
}
\begin{flushright}
Студент: Саржевский Иван
\\[2mm]Группа: P3302
\end{flushright}
\vfill
г. Санкт-Петербург\\[2mm]
2020 г.

\end{center}
\end{titlepage}

\tableofcontents
\newpage

\justify

\section{Цель}

Цель работы – изучить структуру протокольных блоков данных, анализируя реальный
трафик на компьютере студента с помощью бесплатно распространяемой утилиты
\texttt{Wireshark}.

\section{Анализ трафика утилиты \texttt{ping}}

Для анализа трафика, создаваемого утилитой \texttt{ping} был выбран сайт
\textbf{\texttt{www.ias.ru}}.

\figc{ping_headers}{Заголовки протоколов для команды \texttt{ping}.}{p_h}{3.0}

На рисунке \ref{fig:p_h} изображены заголовки различных протоколов, используемых
при передаче запроса.

\subsection{Фрейм}

\figc{ping_frame}{Информация о фрейме команды \texttt{ping}.}{p_f}{3.0}

Структура, представленная на рисунке \ref{fig:p_f}, описывает метаданные
Wireshark для этого запроса - его порядковый номер среди всех записанных,
время прибытия, размер, протокол и цвет выделения в интерфейсе.

\subsection{Ethernet II}

Ethernet II - протокол канального уровня, т.е. описывает передачу данных в
рамках локальной сети. Типичная структура кадра Ethernet II представлена в
таблице \ref{Tab:ethernet_frame}.

\begin{table}[h]
\begin{center}
\begin{small}
  \caption{Структура кадра Ethernet II.}
  \begin{tabular}{| c | c | c | c | c |}
    \hline
    \multicolumn{5}{|c|}{\specialcell[c]{Кадр Ethernet II\\ (от 64-х до 1528-ти байт)}}\\
    \hline
    \multicolumn{3}{|c|}{\specialcell[c]{MAC-заголовок\\ (14 байт)}} & \specialcell[c]{Данные\\ (от 46-ти до 1500 байт)} & --\\
    \hline
    \specialcell[c]{MAC получателя\\ (6 байт)} & \specialcell[c]{MAC отправителя\\ (6 байт)} & \specialcell[c]{Тип протокола\\ (2 байта)} & Данные & \specialcell[c]{CRC\\ (4 байта)}\\
    \hline
  \end{tabular}
  \label{Tab:ethernet_frame}
\end{small}
\end{center}
\end{table}

В данном случае получателем выступает роутер, а отправителем - рабочая машина,
их MAC-адреса записаны в кадр, тип протокола - \texttt{IPv4}, что можно увидеть
на рисунке \ref{fig:p_e}.

\figc{ping_ethernet}{Кадр Ethernet II для \texttt{ping}.}{p_e}{3.0}

\subsection{IPv4}

IPv4 - протокол сетевого уровня. Подробные сведения полях, которые включены в заголовок
протокола, приведены на рисунке \ref{fig:ipv4_h}. Туда включены IP-адреса
отправителя и получателя, длинна заголовка и сообщения, флаги указывающие на
наличие фрагментации данных, промежуточности данного пакета и т. д.

\figc{ipv4_header}{Структура заголовка \texttt{IPv4}.}{ipv4_h}{1.4}

Данные, переданные с использованием протокола \texttt{IPv4} для команды
\texttt{ping} можно увидеть на рисунке \ref{fig:p_ip}.

\figc{ping_ipv4}{Данные пакета \texttt{IPv4} для команды \texttt{ping}.}{p_ip}{3.0}

\subsection{Internet Control Message Protocol (ICMP)}

\figc{icmp_header}{Структура заголовка \texttt{ICMP}.}{icmp_h}{1.4}

Данный протокол сетевого уровня используется для передачи служебных сообщений -
кода ошибки в случае исключительной ситуации, кода запрашиваемой операции и
кода подтверждения в случае удачной передачи. Подробная структура заголовка
\texttt{ICMP} приведена на рисунке \ref{fig:icmp_h}.

\figc{ping_icmp}{Данные \texttt{ICMP} для команды \texttt{ping}.}{ping_icmp}{3.0}

Для команды \texttt{ping} структура \texttt{ICMP} представлена на рисунке
\ref{fig:ping_icmp}.

Структура ответов имеет схожую структуру, отличаться они будут типом \texttt{ICMP},
сменой адресов получателя и отправителя, \texttt{timestamp}'ами.

\subsection{Ответы на вопросы}

\begin{enumerate}
  \item Имеет ли место фрагментация исходного пакета, какое поле на это указывает?\\
    -- Да. Но только в том случае, если размер пакета превышает \texttt{Maximum
    Transmission Unit} (\texttt{MTU}), равный для протокола \texttt{Ethernet II}
    1500 байт. Информация о наличии фрагментации содержится во флаге в заголовке
    \texttt{IPv4}.
  \item Какая информация указывает, является ли фрагмент пакета последним или
    промежуточным?\\
    -- Флаг \texttt{More Fragments} в заголовке \texttt{IPv4}.
  \item Чему равно количество фрагментов при передаче ping-пакетов?\\
    -- \texttt{MTU} равен 1500 байт, пакет включает в себя \texttt{IPv4}-заголовок
    (20 байт), \texttt{ICMP}-заголовок (8 байт), и, непосредственно, данные.
    Это означает, что количество фрагментов равно $\lceil ( s + 20 + 8 ) / 1500 \rceil$,
    где $s$ - аргумент \texttt{-s} команды \texttt{ping}. Зависимость количества
    фрагментов от размера пакета приведена в таблице \ref{Tab:frag}.

    \begin{table}[h]
    \begin{center}
    \begin{small}
      \caption{Количество фрагментов при разных размерах пакета}
      \begin{tabular}{| c | c | c | c | c | c | c | c | c |}
        \hline
        Размер пакета & 100 &	500 & 1000 & 1500 & 2000 & 3000 & 5000 & 10000\\
        \hline
        Кол-во фраг. & 1 & 1 & 1 & 2 & 2 & 3 & 4 & 7\\
        \hline
      \end{tabular}
      \label{Tab:frag}
    \end{small}
    \end{center}
    \end{table}
  \item  Построить график, в котором на оси абсцисс находится размер\_пакета,
    а по оси ординат -- количество фрагментов, на которое был разделён каждый
    \texttt{ping}-пакет.\\
    -- см. рисунок \ref{fig:frag}.
    \figc{frag}{Зависимость количества фрагментов от размера пакета}{frag}{0.8}

  \item Как изменить поле TTL с помощью утилиты ping?\\
    \begin{itemize}
      \item \texttt{Linux} : \texttt{ping -t ttl\_value}
      \item \texttt{Windows} : \texttt{ping -i ttl\_value}
    \end{itemize}
  \item Что содержится в поле данных \texttt{ping}-пакета?\\
    -- В поле данных содержится текущий timestamp, а затем циклически
    повторяющиеся биты от \texttt{00} до \texttt{FF}.\\ \\
    \texttt{0030   00 00 5d 12 04 00 00 00 00 00 10 11 12 13 14 15}\\
    \texttt{0040   16 17 18 19 1a 1b 1c 1d 1e 1f 20 21 22 23 24 25}\\
    \texttt{0050   26 27 28 29 2a 2b 2c 2d 2e 2f 30 31 32 33 34 35}\\
    \texttt{0060   36 37 38 39 3a 3b 3c 3d 3e 3f 40 41 42 43 44 45}\\
    \texttt{                    .       .       .                 }\\
    \texttt{0110   e6 e7 e8 e9 ea eb ec ed ee ef f0 f1 f2 f3 f4 f5}\\
    \texttt{0120   f6 f7 f8 f9 fa fb fc fd fe \textbf{FF 00} 01 02 03 04 05}\\
\end{enumerate}

\section{Анализ трафика утилиты \texttt{traceroute}}

Утилита \texttt{traceroute} отправляет \texttt{UDP}-запросы, постепенно увеличивая
\texttt{ttl}, на 1 каждые 3 запроса.

\figc{trace_headers}{Заголовки протоколов для команды \texttt{traceroute}.}{t_f}{3.0}

На рисунке \ref{fig:t_f} изображены заголовки различных протоколов, используемых при
передаче запроса.

Все протоколы, кроме \texttt{UDP}, были описаны для команды \texttt{ping}.

\figc{trace_ans_headers}{Заголовки протоколов для ответа на команду \texttt{traceroute}.}{t_f_a}{3.0}

На рисунке \ref{fig:t_f_a} изображены заголовки протоколов, используемых при
передаче ответа на запросы команды \texttt{traceroute}.

\subsection{User Datagram Protocol}

\texttt{UDP} - один из протоколов транспортного уровня, который предполагает отсутствие
механизмов установления и поддержки соединения между отправителем и получателем.
Структура \texttt{UDP}-датаграммы представлена в таблице \ref{Tab:udp}.

\begin{table}[h]
\begin{center}
  \caption{Структура \texttt{UDP}-датаграммы.}
  \begin{tabular}{| c | c | c | c |}
    \hline
    Порт отправителя & Порт получателя & Длина датаграммы & Данные\\
    \hline
  \end{tabular}
  \label{Tab:udp}
\end{center}
\end{table}

На рисунке \ref{fig:t_u} можно увидеть содержимое датаграммы для запроса
\texttt{traceroute}.

\figc{trace_udp}{Содержимое датаграммы.}{t_u}{3.0}

\subsection{\texttt{ICMP} в ответах на запросы}

Как уже говорилось ранее, \texttt{ICMP} передает служебную информацию, такую
как сообщения об ошибках. Именно эту его особенность использует утилита
\texttt{traceroute}. Так как мы постепенно увеличиваем \texttt{ttl} начиная
с единицы, каждый маршрутизатор в сети между отправителем и получателем будет
возвращать ошибку \texttt{Time-to-live exceeded} (код 11), и по отправителям этих
сообщений можно судить о маршрутизаторах в сети на пути пакетов. Пример такого
\texttt{ICMP} можно увидеть на рисунке \ref{fig:i_t_e}.

\figc{icmp_ttl_exceeded}{\texttt{ICMP} с сообщением \texttt{ttl-exceeded}}{i_t_e}{3.0}

Если же мы установили достаточно большой \texttt{ttl} для того, чтобы добраться
до получателя, в \texttt{ICMP} с большой вероятностью будет другая ошибка --
\texttt{Destination unreachable} (код 3) с расширением \texttt{Port unreachable}
(код 3), так как \texttt{traceroute} отправляет запросы на случайные порты
получателя. Пример такого \texttt{ICMP} можно увидеть на рисунке \ref{fig:i_d_u}.

\figc{icmp_dest_unreach}{\texttt{ICMP} с сообщением \texttt{Destination unreachable}}{i_d_u}{3.0}

В обоих случаях в ответном сообщении содержится копия исходного.

\subsection{Ответы на вопросы}

\begin{enumerate}
  \item Сколько байт содержится в заголовке IP? Сколько байт содержится в поле данных?\\
    -- \texttt{IPv4} - 20 байт, данных отправляется 32 байта. Если прибавить к
    этому 8 байт заголовка \texttt{UDP} и 14 байт заголовка \texttt{Ethernet II},
    то получим 74 байта, это размер передаваемого сообщения.
  \item Как и почему изменяется поле TTL в следующих друг за другом ICMP-пакетах traceroute?\\
    -- Как уже было упомянуто выше, это позволяет по сообщениям с ошибкой
    \texttt{Time-to-live exceeded} определить все маршрутизаторы, которые
    работали с пакетом по пути до получателя. \texttt{ttl} увеличивается на 1
    каждые 3 отправленных пакета.
  \item  Чем отличаются ICMP-пакеты, генерируемые утилитой tracert, от
    ICMP-пакетов, генерируемых утилитой ping?\\
    -- Использовалась утилита \texttt{traceroute}, которая в отличие от
    \texttt{tracert} отправляет \texttt{UDP} запросы. Однако при использовании
    последней \texttt{ICMP} пакеты от \texttt{ping} отличаются только значением
    \texttt{ttl}.
  \item Чем отличаются полученные пакеты \texttt{ICMP reply} от \texttt{ICMP error} и зачем
    нужны оба этих типа ответов?\\
    -- В \texttt{traceroute} в качестве \texttt{ICMP reply} используется
    \texttt{ICMP error}, но с другим кодом ошибки - \texttt{Destination unreachable (Port unreachable)}.
    Если говорить о \texttt{tracert}, то \texttt{ICMP reply}, получаемые при
    достижении пакетов получателя, ничем не отличаются от аналогичных в утилите
    \texttt{ping}. \texttt{ICMP error} используются для идентификации всех
    маршрутизаторов на пути пакета, это описано в предыдущем вопросе.
  \item Что изменится в работе tracert, если убрать ключ \texttt{-d}? Какой
    дополнительный трафик при этом будет генерироваться?\\
    -- traceroute станет преобразовывать IP адреса маршрутизаторов в их строковые
    адреса, а для этого потребуются дополнительные DNS запросы. ИСПРАВИТЬ
\end{enumerate}

\section{Анализ HTTP-трафика}

Для выполнения этого задания был выбран сайт \texttt{www.ias.ru}.

\figc{http_headers}{Структура запроса}{h_h}{3.0}

На рисунке \ref{fig:h_h} изображена структура запроса, все протоколы, кроме
\texttt{TCP} и \texttt{HTTP} были рассмотрены ранее.

\subsection{Transmission Control Protocol}

Это протокол транспортного уровня, один из основных протоколов передачи данных
в интернете. \texttt{TCP} предполагает надежную передачу потока данных, включает
управление перегрузкой, рукопожатие, передачу данных. Структура \texttt{TCP}-протокола
приведена на рисунке \ref{fig:tcp_s}. Она включает в себя информацию о портах
отправителя и получателя, размер заголовка, а также контрольную сумму и флаги:

\begin{itemize}
  \item \texttt{URG} : указатель важности
  \item \texttt{ACK} : номер подтверждения
  \item \texttt{PSH} : протолкнуть данные, накопившиеся в буфере, в приложение
    пользователя
  \item \texttt{RST} : оборвать соединение, очистить буфер
  \item \texttt{SYN} : синхронизация объектов последовательности
  \item \texttt{FIN} : завершение соединения
\end{itemize}

\figc{tcp_header}{Структура \texttt{TCP}-сегмента.}{tcp_s}{1.7}

Структура переданного сегмента приведена на рисунке \ref{fig:tcp_d}.

\figc{http_tcp}{\texttt{TCP}-сегмент переданного запроса.}{tcp_d}{3.0}

\subsection{Hypertext Transfer Protocol}

Это протокол прикладного уровня для передачи данных. Состоит из стартовой
строки, заголовков и тела сообщения.

В стартовой строке указывается метод запроса, версия запроса, а также путь к
документу. В заголовках передается метаинформация о клиенте или сервере, тело
сообщения содержит непосредственно полезные данные.

Структура первичного \texttt{HTTP}-запроса типа \texttt{GET} показана на рисунке
\ref{fig:h_f_g}.

\figc{http_first_get}{Первичный \texttt{HTTP}-запрос.}{h_f_g}{2.5}

В ответе на первичный запрос приходит \texttt{HTTP}-ответ с кодом 200,
сигнализирующем об успешной обработке запроса, временем последней модификации
запрашиваемого html-документа и самим содержимым требуемого документа. Это
можно увидеть на рисунке \ref{fig:h_f_r}.

\figc{http_first_res}{Первичный \texttt{HTTP}-ответ.}{h_f_r}{2.5}

\newpage

Повторный запрос похож на первичный, за исключением того, что добавляется поле
\texttt{If-Modified-Since} со значением даты, которое мы получили в первичном
ответе. Таким образом, сервер вернет 200 только в случае, если запрашиваемая
html-страница изменялась после переданного времени. Этот запрос можно увидеть
на рисунке \ref{fig:h_s_g}.

\figc{http_second_get}{Повторный \texttt{HTTP}-запрос.}{h_s_g}{2.5}

В ответе на повторный запрос можно увидеть код 304, что означает \texttt{Not Modified}
и сигнализирует о том, что запрашиваемая страница не изменялась с момента времени,
указанного в поле \texttt{If-Modified-Since} запроса, таким образом можно не
передавать её содержимое снова. Это можно увидеть на рисунке \ref{fig:h_s_r}.

\figc{http_second_res}{Повторный \texttt{HTTP}-ответ.}{h_s_r}{2.5}

\newpage

\section{Анализ DNS трафика}

\figc{dns_headers}{Структура запроса}{d_h}{2.5}

На рисунке \ref{fig:d_h} представлена структура запроса. Все протоколы кроме
\texttt{DNS} были рассмотрены ранее.

\subsection{Domain Name System}

Это протокол прикладного уровня, используется для нахождения \texttt{IP}-адреса
по имени хоста. В заголовке определены флаги:

\begin{itemize}
  \item \texttt{Response} : 0 если это запрос, 1 - если ответ
  \item \texttt{Opcode} : 0 - обычный запрос, 1 - обратный запрос, 2 - запрос
    состояния сервера
  \item \texttt{Truncated} : было ли сообщение обрезано из-за чрезмерной длины
  \item \texttt{Recursion Desired} : является ли запрос рекурсивным
  \item \texttt{Z} : зарезервировано (0)
  \item \texttt{Non-authenticated data} : данные из непроверенного источника
\end{itemize}

В поле \texttt{Queries} определена информация о поданных запросах, структура
запроса:

\begin{itemize}
  \item \texttt{Name} : имя домена
  \item \texttt{Type} : тип запроса
  \item \texttt{Class} : класс запроса, обычно выставляется в \texttt{IN},
    для имен хоста, размещенных в интернете
\end{itemize}

\texttt{DNS}-запрос можно увидеть на рисунке \ref{fig:d_rq}.

\figc{dns_req}{\texttt{DNS}-запрос.}{d_rq}{2.5}

В ответ добавляются дополнительные флаги, а именно:

\begin{itemize}
  \item \texttt{Authoritative} : является ли сервер авторитативным
  \item \texttt{Recursion available} : способность сервера обрабатывать
    рекурсивные запросы
  \item \texttt{Answer authenticated} : удостоверенность ответа
  \item \texttt{Reply code} : код ответа
\end{itemize}

Запрос являлся запросом типа \texttt{A}, что значит запрос \texttt{IPv4} адреса,
в ответе данный адрес находится в поле \texttt{Adress}, что можно увидеть на
рисунке \ref{fig:d_rs}.

\figc{dns_res}{\texttt{DNS}-ответ.}{d_rs}{2.5}

\subsection{Ответы на вопросы}

\begin{enumerate}
  \item Почему адрес, на который отправлен DNS-запрос, не совпадает с адресом
    посещаемого сайта?\\
    -- \texttt{DNS}-запрос отправляется не на адрес посещаемого сайта, а на
    адрес \texttt{DNS}-сервера, чтобы узнать \texttt{IP} сайта по его доменному
    имени.
  \item Какие бывают типы DNS-запросов?\\
    -- Итеративный - простой запрос определенному \texttt{DNS}-серверу, рекурсивный -
    \texttt{DNS}-сервер может обратиться рекурсивно к другим \texttt{DNS}-серверам,
    обратный - получение \texttt{IP} по доменному имени.
  \item В какой ситуации нужно выполнять независимые DNS-запросы для получения
    содержащихся на сайте изображений?\\
    -- Когда изображения представлены ссылками на ресурс с другим доменным
    именем, для этого придется делать \texttt{DNS}-запрос чтобы получить \texttt{IP}
    ресурса с изображением по доменному имени.
\end{enumerate}

\section{Анализ \texttt{ARP} трафика}

Для отправки пакетов необходимо определить \texttt{MAC}-адрес роутера. Для этого
компьютер отправляет широковещательное сообщение с \texttt{ARP}-запросом
получателю с \texttt{MAC}-адресом \texttt{FF:FF:FF:FF:FF:FF}, которое принимается
всеми компьютерами в сети, для получения \texttt{MAC}-адреса устройтсва с
\texttt{IP}-адресом \texttt{192.168.0.1}. Как ответ роутер отправляет свой
\texttt{MAC}-адрес, который заносится в \texttt{ARP}-таблицу. Структура запроса
представлена на рисунке \ref{fig:a_h}.

\figc{arp_headers}{Структура запроса.}{a_h}{2.5}

Все протоколы, кроме \texttt{ARP}, были рассмотрены выше.

\subsection{Address Resolution Protocol}

Это протокол сетевого уровня, содержит следующую информацию:

\begin{itemize}
  \item \texttt{Hardware type} : \texttt{HTYPE} -- тип канального протокола
  \item \texttt{Protocol type} : \texttt{PTYPE} -- тип сетевого протокола
  \item \texttt{Hardware length} : \texttt{HLEN} -- длина физического адреса
    в байтах
  \item \texttt{Protocol length} : \texttt{PLEN} -- длина логического адреса
    в байтах
  \item \texttt{Opcode} -- 1 - запрос, 2 - ответ
  \item \texttt{Sender MAC/IP address} -- \texttt{MAC/IP} отправителя
  \item \texttt{Target MAC/IP address} -- \texttt{MAC/IP} получателя (в случае
    запроса игнорируется)
\end{itemize}

Данные, переданные протоколом можно увидеть на рисунке \ref{fig:a_rq}.

\figc{arp_req}{\texttt{ARP}-запрос.}{a_rq}{2.5}

Ответ приходит на адрес, указанный в запросе, поле \texttt{Target MAC address}
заполняется \texttt{MAC}-адресом роутера. Это можно увидеть на рисунке \ref{fig:a_rs}.

\figc{arp_res}{\texttt{ARP}-ответ.}{a_rs}{0.7}

\subsection{Ответы на вопросы}

\begin{enumerate}
  \item Какие МАС-адреса присутствуют в захваченных пакетах ARP-протокола?
    Что означают эти адреса? Какие устройства они идентифицируют?\\
    -- \texttt{c0 b5 d7 64 e6 c5} : компьютер\\
    -- \texttt{cc 32 e5 3d 06 ae} : роутер\\
    -- \texttt{FF FF FF FF FF FF} : широковещательный адрес
  \item Какие МАС-адреса присутствуют в захваченных HTTP-пакетах и что означают
    эти адреса? Что означают эти адреса? Какие устройства они идентифицируют?\\
    -- \texttt{c0 b5 d7 64 e6 c5} : компьютер\\
    -- \texttt{cc 32 e5 3d 06 ae} : роутер
  \item Для чего ARP-запрос содержит IP-адрес источника?\\
    Благодаря тому, что запрос широковещательный, другие устройства в сети
    могут внести в свои \texttt{ARP}-таблицы информацию о \texttt{MAC}-адресе
    роутера.
\end{enumerate}

\end{document}
