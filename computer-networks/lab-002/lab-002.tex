\documentclass[12pt, a4paper]{article}
\usepackage[a4paper, includeheadfoot, mag=1000, left=2cm, right=1.5cm, top=1.5cm, bottom=1.5cm, headsep=0.8cm, footskip=0.8cm]{geometry}
% Fonts
\usepackage{fontspec, unicode-math}
\setmainfont[Ligatures=TeX]{CMU Serif}
\setmonofont{CMU Typewriter Text}
\usepackage[english, russian]{babel}
% Indent first paragraph
\usepackage{indentfirst}
\setlength{\parskip}{5pt}
% Diagrams
\usepackage{graphicx}
\usepackage{float}
% Page headings
\usepackage{fancyhdr}
\pagestyle{fancy}
\renewcommand{\headrulewidth}{0pt}
\setlength{\headheight}{16pt}
%\newfontfamily\namefont[Scale=1.2]{Gloria Hallelujah}
\fancyhead{}
\usepackage{ragged2e}

\usepackage{multirow}

\usepackage{listings}
\lstdefinestyle{lablisting}{
  basicstyle=\scriptsize\ttfamily,
  numbers=left,
  stepnumber=1,
  otherkeywords={EOF, O\_RDONLY, STDIN\_FILENO, STDOUT\_FILENO, STDERR\_FILENO},
  numbersep=10pt,
  showspaces=false,
  showstringspaces=false
}

\newcommand{\specialcell}[2][l]{%
  \begin{tabular}[#1]{@{}c@{}}#2\end{tabular}}

\begin{document}

% Title page
\begin{titlepage}
\begin{center}

\textsc{Национальный исследовательский университет ИТМО\\[4mm]
Факультет программной инженерии и компьютерной техники}
\vfill
\textbf{Учебно-исследовательская работа №2\\[4mm]
по дисципение Сети ЭВМ и телекоммуникации\\[16mm]
}
\begin{flushright}
Студент: Саржевский Иван
\\[2mm]Группа: P3302
\end{flushright}
\vfill
г. Санкт-Петербург\\[2mm]
2020 г.

\end{center}
\end{titlepage}

\tableofcontents
\newpage

\justify

\section{Цель}

Исследование влияния свойств канала связи на качество передачи сигналов при
различных методах физического и логического кодирования, используемых в цифровых
сетях передачи данных.

\section{Задание}

Для заданного исходного сообщения и заданных методов кодирования выполнить
исследование качества передачи физических сигналов в зависимости от уровня
шумов в канале связи, уровня рассинхронизации передатчика и приемника и уровня
граничного напряжения. Сравнить расссматриваемые методы кодирования, выбрать и
обосновать наилучший метод для передачи исходного сообщения по реальному каналу
связи.

\section{Ход работы}

\subsection{Исходные данные}

\begin{tabular}{ l l }
  \textit{Сообщение:} & \texttt{Сарж} \\
  \textit{Hex-код:} & \texttt{D1 E0 F0 E6} \\
  \textit{Bin-код:} & \texttt{11010001 11100000 11110000 11100110} \\
  \textit{Длина:} & 4 байта (32 бита)
\end{tabular}

\subsection{Результаты исследования}

\newpage

\begin{table}[h]
\caption{Результаты исследований}
\begin{center}
\begin{tabular}{| c | c | c | c | c | c | c | c | c |}
  \hline
  \multicolumn{3}{|c|}{\multirow{2}{*}{HEX: \texttt{D1 E0 F0 E6}}} & \multicolumn{6}{|c|}{Метод кодирования}\\
  \cline{4-9}
  \multicolumn{3}{|c|}{} & NRZ & RZ & AMI & M-II & 4B/5B & Scramb\\
  \hline
  \multirow{4}{*}{\specialcell[c]{
      Полоса\\
      пропускания\\
      идеального\\
      канала связи
    }} & \multirow{2}{*}{Гармоники} & мин & 6 & 6 & 0 & 0 & 0 & 0\\
  \cline{3-9}
  & & макс & 28 & 56 & 0 & 0 & 0 & 0\\
  \cline{2-9}
  & \multirow{2}{*}{Частоты, МГц} & мин & 0.9 & 0.9 & 0 & 0 & 0 & 0\\
  \cline{3-9}
  & & макс & 4.4 & 8.8 & 0 & 0 & 0 & 0\\
  \hline
  \multicolumn{3}{|c|}{\specialcell[c]{
      Минимальная полоса пропускания\\
      идеального канала связи
    }} & 3.5 & 7.9 & 0 & 0 & 0 & 0\\
  \hline
  \multicolumn{2}{|c|}{Уровень шума} & макс & 0.02 & 0.07 & 0 & 0 & 0 & 0\\
  \hline
  \multicolumn{2}{|c|}{\specialcell[c]{
      Уровень\\
      рассинхронизации
    }} & макс & 0.1 & 0.27 & 0 & 0 & 0 & 0\\
  \hline
  \multicolumn{2}{|c|}{\specialcell[c]{
      Уровень граничного\\
      напряжения
    }} & макс & 0.51 & 0.58 & 0 & 0 & 0 & 0\\
  \hline
  \multicolumn{3}{|c|}{\specialcell[c]{
      \% ошибок при max уровнях и мини-\\
      мальной полосе пропускания КС
    }} & 0.88 & 2.04 & 0 & 0 & 0 & 0\\
  \hline
  \multicolumn{2}{|c|}{Уровень шума} & ср & 0 & 0 & 0 & 0 & 0 & 0\\
  \hline
  \multicolumn{2}{|c|}{\specialcell[c]{
      Уровень\\
      рассинхронизации
    }} & ср & 0 & 0 & 0 & 0 & 0 & 0\\
  \hline
  \multicolumn{2}{|c|}{\specialcell[c]{
      Уровень граничного\\
      напряжения
    }} & ср & 0 & 0 & 0 & 0 & 0 & 0\\
  \hline
  \multirow{4}{*}{\specialcell[c]{
      Полоса\\
      пропускания\\
      идеального\\
      канала связи
    }} & \multirow{2}{*}{Гармоники} & мин & 0 & 0 & 0 & 0 & 0 & 0\\
  \cline{3-9}
  & & макс & 0 & 0 & 0 & 0 & 0 & 0\\
  \cline{2-9}
  & \multirow{2}{*}{Частоты, МГц} & мин & 0 & 0 & 0 & 0 & 0 & 0\\
  \cline{3-9}
  & & макс & 0 & 0 & 0 & 0 & 0 & 0\\
  \hline
  \multicolumn{3}{|c|}{\specialcell[c]{
      Требуемая полоса пропускания\\
      реального канала связи
    }} & 0 & 0 & 0 & 0 & 0 & 0\\
  \hline
\end{tabular}
\end{center}
\label{Tab:res}
\end{table}

\section{Вывод}

\end{document}
