\documentclass[12pt, a4paper]{article}
\usepackage[a4paper, includeheadfoot, mag=1000, left=2cm, right=1.5cm, top=1.5cm, bottom=1.5cm, headsep=0.8cm, footskip=0.8cm]{geometry}
% Fonts
\usepackage{fontspec, unicode-math}
\setmainfont[Ligatures=TeX]{CMU Serif}
\setmonofont{CMU Typewriter Text}
\usepackage[english, russian]{babel}
% Indent first paragraph
\usepackage{indentfirst}
\setlength{\parskip}{5pt}
% Diagrams
\usepackage{graphicx}
\usepackage{float}
% Page headings
\usepackage{fancyhdr}
\pagestyle{fancy}
\renewcommand{\headrulewidth}{0pt}
\setlength{\headheight}{16pt}
%\newfontfamily\namefont[Scale=1.2]{Gloria Hallelujah}
\fancyhead{}
% Python
\usepackage{pythontex}
\usepackage{ragged2e}

\usepackage{listings}
\lstdefinestyle{lablisting}{
  basicstyle=\scriptsize\ttfamily,
  numbers=left,
  stepnumber=1,
  otherkeywords={EOF, O\_RDONLY, STDIN\_FILENO, STDOUT\_FILENO, STDERR\_FILENO},
  numbersep=10pt,
  showspaces=false,
  showstringspaces=false
}

\graphicspath{ {images/} }

\newcommand{\specialcell}[2][l]{%
  \begin{tabular}[#1]{@{}l@{}}#2\end{tabular}}

\newcommand{\bandwidthEntry}[2]{
  \hline
  \multicolumn{2}{|c|}{Пропускная способность: #1 Mbps} \\
  \hline
  Частота основной гармоники сигнала \texttt{11111...} & \py{int(f_0[#1] * #2)} Гц \\
  Верхняя граница частот в передаваемом сообщении & \py{int(max_mult * f_0[#1])} Гц \\
  Нижняя граница частот в передаваемом сообщении & \py{int(f_0[#1] / max_count)} Гц \\
  Необходимая полоса пропускания & \py{int(f_0[#1] / max_count)} : \py{int(max_mult * f_0[#1])} Гц \\
}

\begin{document}

\begin{pycode}
from utils import *
\end{pycode}

% Title page
\begin{titlepage}
\begin{center}

\textsc{Национальный исследовательский университет ИТМО\\[4mm]
Факультет программной инженерии и компьютерной техники}
\vfill
\textbf{Учебно-исследовательская работа №1\\[4mm]
по дисципение Сети ЭВМ и телекоммуникации\\[16mm]
}
\begin{flushright}
Студент: Саржевский Иван
\\[2mm]Группа: P3302
\end{flushright}
\vfill
г. Санкт-Петербург\\[2mm]
2020 г.

\end{center}
\end{titlepage}

\tableofcontents
\newpage

\justify

\section{Цель}

Изучение методов логического и физического кодирования, используемых в цифровых
сетях передачи данных.

\section{Задание}

Выполнить логическое и физическое кодирование исходного сообщения в соответствии
с заданными методами кодирования, провести сравнительный анализ рассматриваемых
методов кодирования, выбрать и обосновать наилучший метод для передачи исходного
сообщения.

\section{Формирование сообщения}

\begin{tabular}{ l l }
  \textit{Сообщение:} & \texttt{Саржевский И.А.} \\
  \textit{Hex-код:} & \texttt{D1 E0 F0 E6 E5 E2 F1 EA E8 E9 20 C8 2E C0 2E} \\
  \textit{Bin-код:} & \specialcell[l]{
    \texttt{11010001 11100000 11110000 11100110 11100101 11100010 11110001} \\
    \texttt{11101010 11101000 11101001 00100000 11001000 00101110 11000000} \\
    \texttt{00101110}
  } \\
  \textit{Длина:} & 15 байт (120 бит)
\end{tabular}

\section{Физическое кодирование}

\subsection{Выделение основной частоты}

Определим частоту основной гармоники бесконечного сигнала \texttt{10101010...}
как базовую частоту $f_0$. В таком случае, частоту основной гармоники сигнала
с повторением произвольного количества нулей и единиц удобно представить в
виде $f = f_0 / x$, где $x$ - длина последовательности нулей и единиц (например,
очевидно что частота основной гармоники сигнала \texttt{110011001100...} равна
$f_0 / 2$).

Введение такой переменной позволяет нам записать ряд Фурье для бесконечного
сигнала с повторением произвольного количества нулей и единиц таким образом:

$$\sum_y (A_0 / y) \sin (2 \pi \frac{y f_0}{x} t)$$

, где $y \in \{ 1, 3, 5, 7 \}$, а $x$ - длина периода повторяющихся символов.

Определим $f_0$ для разных значений пропускной способности. Она будет равна
единице, деленной на период основной гармоники, равный времени передачи двух
бит сообщения. Таким образом, для пропускной способности $x$:

\newpage

\noindent $B_t = 1 / x$ \\
$T_0 = 2 B_t$ \\
$f_0 = 1 / T_0$ \\

\begin{tabular}{ c | c }
  Пропускная способность, Mbps & Базовая частота $f_0$, Гц \\
  10 & $0.5 * 10^7$ \\
  100 & $0.5 * 10^8$ \\
  1000 & $0.5 * 10^9$
\end{tabular}

\subsection{Потенциальный код NRZ}

\begin{figure}[h]
  \begin{center}
    \includegraphics{nrz}
    \caption{Кодирование первых четырех байт потенциальным кодом NRZ}
    \label{fig:nrz}
  \end{center}
\end{figure}

\begin{pycode}
nrz_string = '11010001111000001111000011100110111001011110001011110001111010'\
  '1011101000111010010010000011001000001011101100000000101110'
[max_count, max_char] = longest_substing(nrz_string)
max_mult = 7
\end{pycode}

Из рис. \ref{fig:nrz} легко понять, что $T_{min}$ достигается при кодировании чередующихся
сигналов, из чего следует, что $T_{min} = T_0$ и $f_{max} = f_0$. Для определения
$T_{max}$ найдем участок с минимальным чередованием. Для данного вида кодирования
такой участок достигается при передаче последовательности из \py{max_count}
символов \texttt{\py{max_char}},
следовательно $T_{max} = \py{max_count} T_0$, а $f_{min} = \frac{f_0}{\py{max_count}}$.

Разложение в ряд Фурье для сигнала представляющего из себя последовательность
чередующихся единиц и нулей будет иметь вид:

$$A_0 \sin(2 \pi f_0 t) + (A_0 / 3) \sin(2 \pi 3 f_0 t) +
  (A_0 / 5) \sin(2 \pi 5 f_0 t) + (A_0 / 7) \sin(2 \pi 7 f_0 t)$$

Аналогичное разложение для сигнала с чередующимися последовательностями из
\py{max_count} единиц и нулей будет иметь вид:

$$A_0 \sin(2 \pi \frac{f_0}{\py{max_count}} t) + (A_0 / 3) \sin(2 \pi \frac{3 f_0}{\py{max_count}} t) +
  (A_0 / 5) \sin(2 \pi \frac{5 f_0}{\py{max_count}} t) + (A_0 / 7) \sin(2 \pi \frac{7 f_0}{\py{max_count}} t)$$

Получаем спектр частот $\frac{f_0}{\py{max_count}}...7 f_0$.

\newpage

\begin{tabular}{| c | c |}
  \bandwidthEntry{10}{0}
  \bandwidthEntry{100}{0}
  \bandwidthEntry{1000}{0}
  \hline
\end{tabular}

\subsubsection*{Определение среднего значения чаcтоты передаваемого сообщения}

\begin{pycode}
occurrence = get_occurrence_counts(nrz_string)
build_occurrence_table(occurrence)

mes_len = len(nrz_string)
print(get_f_mean_math(occurrence, mes_len))
get_f_mean_table(occurrence, mes_len)
f_mean_nrz = round(((max_mult * f_0[1000]) - (f_0[1000] / max_count)) / 1000000000, 2)
\end{pycode}

\subsection{Манчестерское кодирование}

\begin{figure}[h]
  \begin{center}
    \includegraphics{manchester}
    \caption{Кодирование первых четырех байт Манчестерским кодом}
    \label{fig:man}
  \end{center}
\end{figure}

\begin{pycode}
man_string = '110100011110000011110000111001101110010111100010111100011110101'\
  '011101000111010010010000011001000001011101100000000101110'
max_count = 1
max_mult = 14
\end{pycode}

Из рис. \ref{fig:man} легко понять, что $T_{min}$ достигается при кодировании последовательных
одинаковых бит (\texttt{00} или \texttt{11}), при этом $T_{min} = T_0 / 2$ 
и $f_{max} = 2 f_0$. $T_{max}$ возникает в следствие кодирования чередующихся
бит (\texttt{01} и \texttt{10}) и равен $f_0$. Сигнал вида \texttt{11111...} или
\texttt{00000...} аналогичен последовательности сигналов \texttt{11}, следовательно
частота его основной гармоники равна $f_{max}$.

Разложение в ряд Фурье для сигнала представляющего из себя последовательность
единиц или нулей будет иметь вид:

$$A_0 \sin(2 \pi 2 f_0 t) + (A_0 / 3) \sin(2 \pi 6 f_0 t) +
  (A_0 / 5) \sin(2 \pi 10 f_0 t) + (A_0 / 7) \sin(2 \pi 14 f_0 t)$$

Аналогичное разложение для сигнала с чередующимися последовательностями из
единиц и нулей будет иметь вид:

$$A_0 \sin(2 \pi f_0 t) + (A_0 / 3) \sin(2 \pi 3 f_0 t) +
  (A_0 / 5) \sin(2 \pi 5 f_0 t) + (A_0 / 7) \sin(2 \pi 7 f_0 t)$$

Получаем спектр частот $f_0 ... 14 f_0$

\begin{tabular}{| c | c |}
  \bandwidthEntry{10}{2}
  \bandwidthEntry{100}{2}
  \bandwidthEntry{1000}{2}
  \hline
\end{tabular}

\subsubsection*{Определение среднего значения чаcтоты передаваемого сообщения}

\begin{pycode}
#occurrence = get_certain_tokens_occerrence(man_string, [['11', '00'], ['10', '01']])
occurrence = f_mean_man_nz(man_string)
build_occurrence_table(occurrence, True, [2, 1])

mes_len = len(man_string)
print(get_f_mean_math(occurrence, mes_len))
get_f_mean_table(occurrence, mes_len, True, [2, 1])
f_mean_man = round(((max_mult * f_0[1000]) - (f_0[1000] / max_count)) / 1000000000, 2)
\end{pycode}

\newpage

\subsection{Импульсный код RZ}

\begin{figure}[h]
  \begin{center}
    \includegraphics{rz}
    \caption{Кодирование первых четырех байт импульсным кодом RZ}
    \label{fig:rz}
  \end{center}
\end{figure}

\begin{pycode}
nz_string = '11010001111000001111000011100110111001011110001011110001111010101'\
  '1101000111010010010000011001000001011101100000000101110'
max_count = 1
max_mult = 14
\end{pycode}

Из рис. \ref{fig:rz} легко понять, что $T_{min}$ достигается при кодировании последовательных
одинаковых бит (\texttt{00} или \texttt{11}), при этом $T_{min} = T_0 / 2$ 
и $f_{max} = 2 f_0$. $T_{max}$ возникает в следствие кодирования чередующихся
бит (\texttt{01} и \texttt{10}) и равен $f_0$. Сигнал вида \texttt{11111...} или
\texttt{00000...} аналогичен последовательности сигналов \texttt{11}, следовательно
частота его основной гармоники равна $f_{max}$.

Разложение в ряд Фурье для сигнала представляющего из себя последовательность
единиц или нулей будет иметь вид:

$$A_0 \sin(2 \pi 2 f_0 t) + (A_0 / 3) \sin(2 \pi 6 f_0 t) +
  (A_0 / 5) \sin(2 \pi 10 f_0 t) + (A_0 / 7) \sin(2 \pi 14 f_0 t)$$

Аналогичное разложение для сигнала с чередующимися последовательностями из
единиц и нулей будет иметь вид:

$$A_0 \sin(2 \pi f_0 t) + (A_0 / 3) \sin(2 \pi 3 f_0 t) +
  (A_0 / 5) \sin(2 \pi 5 f_0 t) + (A_0 / 7) \sin(2 \pi 7 f_0 t)$$

Получаем спектр частот $f_0 ... 14 f_0$

\begin{tabular}{| c | c |}
  \bandwidthEntry{10}{2}
  \bandwidthEntry{100}{2}
  \bandwidthEntry{1000}{2}
  \hline
\end{tabular}

\subsubsection*{Определение среднего значения чаcтоты передаваемого сообщения}


\begin{pycode}
#occurrence = get_certain_tokens_occerrence(man_string, [['11', '00'], ['10', '01']])
occurrence = f_mean_man_nz(man_string)
build_occurrence_table(occurrence, True, [2, 1])

mes_len = len(man_string)
print(get_f_mean_math(occurrence, mes_len))
get_f_mean_table(occurrence, mes_len, True, [2, 1])
f_mean_rz = round(((max_mult * f_0[1000]) - (f_0[1000] / max_count)) / 1000000000, 2)
\end{pycode}

\subsection{Потенциальный код 2BQ1}

\begin{figure}[h]
  \begin{center}
    \includegraphics{pam}
    \caption{Кодирование первых четырех байт потенциальным кодом 2BQ1}
    \label{fig:2bq1}
  \end{center}
\end{figure}

\begin{pycode}
pam_string = '1101000111100000111100001110011011100101111000101111000111101010'\
  '11101000111010010010000011001000001011101100000000101110'
max_count = 14
max_mult = 3.5
\end{pycode}

$T_{max}$ достигается на участке \texttt{11101010111010} и равен $14 T_0$,
следовательно $f_{min} = f_0 / 14$. Как видно на рис. \ref{fig:2bq1}, $T_{min}$
достигается на участках с чередованием \texttt{11} и \texttt{00} и равна $2 T_0$,
следовательно $f_{max} = f_0 / 2$. Сигналы вида \texttt{111111...} и
\texttt{000000...} выраждаются в прямую линию, поэтому частота в этом случае равна 0.

Разложение в ряд Фурье для сигнала, имеющего период $T_{min}$ будет иметь вид:

$$A_0 \sin(2 \pi \frac{f_0}{2} t) + (A_0 / 3) \sin(2 \pi 3 \frac{f_0}{2} t) +
  (A_0 / 5) \sin(2 \pi 5 \frac{f_0}{2} t) + (A_0 / 7) \sin(2 \pi 7 \frac{f_0}{2} t)$$

Аналогичное разложение для сигнала с периодом $T_{max}$:

$$A_0 \sin(2 \pi \frac{f_0}{14} t) + (A_0 / 3) \sin(2 \pi 3 \frac{f_0}{14} t) +
  (A_0 / 5) \sin(2 \pi 5 \frac{f_0}{14} t) + (A_0 / 7) \sin(2 \pi 7 \frac{f_0}{14} t)$$

Получаем спектр частот $f_0 / 14 ... 3.5 f_0$

\begin{tabular}{| c | c |}
  \bandwidthEntry{10}{0}
  \bandwidthEntry{100}{0}
  \bandwidthEntry{1000}{0}
  \hline
\end{tabular}

\subsubsection*{Определение среднего значения чаcтоты передаваемого сообщения}

\begin{center}
\begin{tabular}{| c | c | c | c |}
  \hline
  Частота & Участки (кол-во символов) & Кол-во бит. интервалов & Отношение к $f_0$\\
  \hline
  $f_1$ & 2 & 6 & $f_0 / 2$\\
  $f_2$ & 4 & 20 & $f_0 / 4$\\
  $f_3$ & 6 & 6 & $f_0 / 6$\\
  \hline
\end{tabular}
\end{center}

$$f_{mean} = 6 / 32 * f_1 + 20 / 32 * f_2 + 6 / 32 * f_3$$

\begin{center}
\begin{tabular}{| c | c |}
  \hline
  Пропускная способность, Mbps & $f_{mean}$, Гц\\
  \hline
  10 & 1406250\\
  100 & 14062500\\
  1000 & 140625000\\
  \hline
\end{tabular}
\end{center}

\subsection{Сравнительный анализ физического кодирования}

\begin{center}
\begin{tabular}{| c | c | c | c |}
  \hline
  Вид & Стоимость & Самосинхронизация & Полоса пропускания\\
  \hline
  Манчестр. & Мин. & Да & \py{f_mean_man} ГГц\\
  NRZ & Мин. & Нет & \py{f_mean_nrz} ГГц\\
  RZ & Сред. & Да & \py{f_mean_rz} ГГц\\
  2BQ1 & Макс. & Нет & 1.7 ГГц\\
  \hline
\end{tabular}
\end{center}

\begin{itemize}
  \item Манчестерское: Не требует синхронизации, требует более широкую полосу
    пропускания чем 2BQ1 и NRZ.
  \item NRZ: Средняя ширина необходимой полосы пропускания, вероятность ошибок
    при передаче длинных последовательностей нулей и единиц.
  \item RZ: Обладает самосинхронизацией, однако дороже в реализации предыдущих
    из-за трех уровней сигнала и широкой необходимой полосы пропускания.
  \item 2BQ1: Самая узкая полоса пропускания, четыре уровня сигнала делают его
    самым дорогим видом кодирования из представленных.
\end{itemize}

Можно сделать вывод, что оптимальным методом кодирования для данного сообщения
будет манчестерский, так как не смотря на широкую необходимую полосу пропускания,
он обладает свойством самосинхронизации и минимальной стоимостью.

\section{Логическое кодирование}

\subsection{4B/5B}

Таблица кодирования 4B/5B:

\begin{pycode}
build_4b5b_table()
m = '1101000111100000111100001110011011100101111000101111000111101010111010001'\
  '11010010010000011001000001011101100000000101110'
m_4b5b = convert_4b5b(m)
\end{pycode}

Исходное сообщение:\\
\py{format_message(m)}

Закодированное сообщение:\\
\py{format_message(m_4b5b)}

Длина закодированного: \py{len(m_4b5b)}; Длина исходного: \py{len(m)}.
Избыточность: \py{int((len(m_4b5b) - len(m)) / len(m) * 100)}\%.

\subsubsection*{Физическое кодирование}

Для кодирования полученного сообщения будем использовать потенциальный код NRZ.

\begin{figure}[h]
  \begin{center}
    \includegraphics{nrz_4b5b}
    \caption{Кодирование первых четырех байт потенциальным кодом NRZ}
    \label{fig:nrz4b5b}
  \end{center}
\end{figure}

\begin{pycode}
[max_count, max_char] = longest_substing(m_4b5b)
mes_len = len(m_4b5b)
max_mult = 7
\end{pycode}

Из рис. \ref{fig:nrz4b5b} легко понять, что $T_{min}$ достигается при кодировании чередующихся
сигналов, из чего следует, что $T_{min} = T_0$ и $f_{max} = f_0$. Для определения
$T_{max}$ найдем участок с минимальным чередованием. Для данного вида кодирования
такой участок достигается при передаче последовательности из \py{max_count}
символов \texttt{\py{max_char}},
следовательно $T_{max} = \py{max_count} T_0$, а $f_{min} = \frac{f_0}{\py{max_count}}$.

Разложение в ряд Фурье для сигнала представляющего из себя последовательность
чередующихся единиц и нулей будет иметь вид:

$$A_0 \sin(2 \pi f_0 t) + (A_0 / 3) \sin(2 \pi 3 f_0 t) +
  (A_0 / 5) \sin(2 \pi 5 f_0 t) + (A_0 / 7) \sin(2 \pi 7 f_0 t)$$

Аналогичное разложение для сигнала с чередующимися последовательностями из
\py{max_count} единиц и нулей будет иметь вид:

$$A_0 \sin(2 \pi \frac{f_0}{\py{max_count}} t) + (A_0 / 3) \sin(2 \pi \frac{3 f_0}{\py{max_count}} t) +
  (A_0 / 5) \sin(2 \pi \frac{5 f_0}{\py{max_count}} t) + (A_0 / 7) \sin(2 \pi \frac{7 f_0}{\py{max_count}} t)$$

Получаем спектр частот $\frac{f_0}{\py{max_count}}...7 f_0$.

\begin{tabular}{| c | c |}
  \bandwidthEntry{10}{0}
  \bandwidthEntry{100}{0}
  \bandwidthEntry{1000}{0}
  \hline
\end{tabular}

\subsubsection*{Определение среднего значения чаcтоты передаваемого сообщения}

\begin{pycode}
occurrence = get_occurrence_counts(m_4b5b)
build_occurrence_table(occurrence)

mes_len = len(m_4b5b)
print(get_f_mean_math(occurrence, mes_len))
get_f_mean_table(occurrence, mes_len)
f_mean_4b5b = round(((max_mult * f_0[1000]) - (f_0[1000] / max_count)) / 1000000000, 2)
\end{pycode}

\subsection{Скремблирование 5-7}

Для скремблирования был выбран полином $A_i \oplus B_{i-5} \oplus B_{i-7}$,
так как по сравнению с полиномом 3-5 интервалы меньшей длины занимают больше
битовых интервалов.

\begin{pycode}
scrembling_str = scrembling57(m)
\end{pycode}

Результат скремблирования:\\
\py{format_message(scrembling_str)}

\subsubsection*{Физическое кодирование}

Для кодирования полученного сообщения будем использовать потенциальный код NRZ.

\begin{figure}[h]
  \begin{center}
    \includegraphics{nrz_scr}
    \caption{Кодирование первых четырех байт потенциальным кодом NRZ}
    \label{fig:nrzscr}
  \end{center}
\end{figure}

\begin{pycode}
[max_count, max_char] = longest_substing(scrembling_str)
mes_len = len(scrembling_str)
max_mult = 7
\end{pycode}

Из рис. \ref{fig:nrzscr} легко понять, что $T_{min}$ достигается при кодировании чередующихся
сигналов, из чего следует, что $T_{min} = T_0$ и $f_{max} = f_0$. Для определения
$T_{max}$ найдем участок с минимальным чередованием. Для данного вида кодирования
такой участок достигается при передаче последовательности из \py{max_count}
символов \texttt{\py{max_char}},
следовательно $T_{max} = \py{max_count} T_0$, а $f_{min} = \frac{f_0}{\py{max_count}}$.

Разложение в ряд Фурье для сигнала представляющего из себя последовательность
чередующихся единиц и нулей будет иметь вид:

$$A_0 \sin(2 \pi f_0 t) + (A_0 / 3) \sin(2 \pi 3 f_0 t) +
  (A_0 / 5) \sin(2 \pi 5 f_0 t) + (A_0 / 7) \sin(2 \pi 7 f_0 t)$$

Аналогичное разложение для сигнала с чередующимися последовательностями из
\py{max_count} единиц и нулей будет иметь вид:

$$A_0 \sin(2 \pi \frac{f_0}{\py{max_count}} t) + (A_0 / 3) \sin(2 \pi \frac{3 f_0}{\py{max_count}} t) +
  (A_0 / 5) \sin(2 \pi \frac{5 f_0}{\py{max_count}} t) + (A_0 / 7) \sin(2 \pi \frac{7 f_0}{\py{max_count}} t)$$

Получаем спектр частот $\frac{f_0}{\py{max_count}}...7 f_0$.

\begin{tabular}{| c | c |}
  \bandwidthEntry{10}{0}
  \bandwidthEntry{100}{0}
  \bandwidthEntry{1000}{0}
  \hline
\end{tabular}

\subsubsection*{Определение среднего значения чаcтоты передаваемого сообщения}

\begin{pycode}
occurrence = get_occurrence_counts(scrembling_str)
build_occurrence_table(occurrence)

mes_len = len(scrembling_str)
print(get_f_mean_math(occurrence, mes_len))
get_f_mean_table(occurrence, mes_len)
f_mean_scr = round(((max_mult * f_0[1000]) - (f_0[1000] / max_count)) / 1000000000, 2)
\end{pycode}

\subsection{Сравнение логических методов кодирования}

\begin{center}
\begin{tabular}{| c | c | c | c | c | c |}
  \hline
  Тип & Ст-ть & \specialcell[c]{
    В-ть\\
    рассхинхрон.
  } & Помехоустойчивость & \specialcell[c]{
    Полоса\\
    пропускания
  } & Избыточность \\
  \hline
  4B5B & min & сред & Запрещен. комбинации & \py{f_mean_4b5b} ГГц & 25\% \\
  \hline
  Скремблинг & max & сред & Нет & \py{f_mean_scr} & 0\% \\
  \hline
\end{tabular}
\end{center}

Для данного сообщения метод кодирования 4B5B будет более удачным. Не смотря на
то, что он имеет 25\% избыточности (что с учетом размера сообщения совсем не много),
он дешевле в реализации и обладает свойством выявления ошибок за счет запрещенных
комбинаций, помимо этого он обладает немного меньшей необходимой шириной пропускания.

\end{document}
