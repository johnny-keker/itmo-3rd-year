\documentclass[12pt, a4paper]{article}
\usepackage[a4paper, includeheadfoot, mag=1000, left=2cm, right=1.5cm, top=1.5cm, bottom=1.5cm, headsep=0.8cm, footskip=0.8cm]{geometry}
% Fonts
\usepackage{fontspec, unicode-math}
\setmainfont[Ligatures=TeX]{CMU Serif}
\setmonofont{CMU Typewriter Text}
\usepackage[english, russian]{babel}
% Indent first paragraph
\usepackage{indentfirst}
\setlength{\parskip}{5pt}
% Diagrams
\usepackage{graphicx}
\usepackage{float}
% Page headings
\usepackage{fancyhdr}
\pagestyle{fancy}
\renewcommand{\headrulewidth}{0pt}
\setlength{\headheight}{16pt}
%\newfontfamily\namefont[Scale=1.2]{Gloria Hallelujah}
\fancyhead{}
\usepackage{ragged2e}

\usepackage{multirow}

\usepackage{listings}
\lstdefinestyle{lablisting}{
  basicstyle=\scriptsize\ttfamily,
  numbers=left,
  stepnumber=1,
  otherkeywords={EOF, O\_RDONLY, STDIN\_FILENO, STDOUT\_FILENO, STDERR\_FILENO},
  numbersep=10pt,
  showspaces=false,
  showstringspaces=false
}

\graphicspath{ {images/} }

\newcommand{\specialcell}[2][l]{%
  \begin{tabular}[#1]{@{}c@{}}#2\end{tabular}}

\newcommand{\figc}[4]{
  \begin{figure}[H]
  \begin{center}
    \includegraphics[scale=#4]{#1}
    \caption{#2}
    \label{fig:#3}
  \end{center}
  \end{figure}
}

\begin{document}

% Title page
\begin{titlepage}
\begin{center}

\textsc{Национальный исследовательский университет ИТМО\\[4mm]
Факультет программной инженерии и компьютерной техники}
\vfill
\textbf{Учебно-исследовательская работа №5\\[4mm]
по дисципение Сети ЭВМ и телекоммуникации\\[4mm]
Технологии QoS в компьютерных сетях\\[16mm]
}
\begin{flushright}
Студент: Саржевский Иван
\\[2mm]Группа: P3302
\end{flushright}
\vfill
г. Санкт-Петербург\\[2mm]
2020 г.

\end{center}
\end{titlepage}

\tableofcontents
\newpage

\justify

\section{Цель}

Изучение эффективности приоритезации трафика для управления качеством
обслуживания (Quality of Service, QoS) в компьютерных сетях. 

\section{Исходные данные}

\begin{center}
\begin{tabular}{c c}
  \texttt{S} & \texttt{10} Кб\\
  \texttt{N} & \texttt{4} Кб\\
  \texttt{K} & \texttt{2}\\
\end{tabular}
\end{center}

\section{Ход работы}

С использованием программы \texttt{Wireshark} было захвачено по 10000
пакетов для трафика \texttt{Skype} и ВПЗ. Для трансляции ВПЗ был выбран
сайт \texttt{webinar.ru}. Примеры захваченного трафика можно увидеть на
рисунках \ref{fig:s_t} и \ref{fig:v_t}.

\figc{skype_traffic}{\texttt{Skype}-трафик.}{s_t}{1.0}
\figc{vod_traffic}{ВПЗ-трафик.}{v_t}{0.8}

По полученным данным были построены фукции распределения для интервалов
между пакетами и размеров пакетов для каждого вида трафика. Полученные
функции распределения представлены на рисунках \ref{fig:s_f}-\ref{fig:v_s}.

\figc{charts/skype_f_time}{Функция распределения для интервалов между пакетами \texttt{Skype}}{s_f}{0.75}
\figc{charts/skype_f_size}{Функция распределения для размеров пакетов \texttt{Skype}}{s_s}{0.75}
\figc{charts/vod_f_time}{Функция распределения для интервалов между пакетами ВПЗ}{v_f}{0.75}
\figc{charts/vod_f_size}{Функция распределения для размеров пакетов ВПЗ}{v_s}{0.75}

\end{document}
