\documentclass[12pt, a4paper]{article}
\usepackage[a4paper, includeheadfoot, mag=1000, left=2cm, right=1.5cm, top=1.5cm, bottom=1.5cm, headsep=0.8cm, footskip=0.8cm]{geometry}
% Fonts
\usepackage{fontspec, unicode-math}
\setmainfont[Ligatures=TeX]{CMU Serif}
\setmonofont{CMU Typewriter Text}
\usepackage[english, russian]{babel}
% Indent first paragraph
\usepackage{indentfirst}
\setlength{\parskip}{5pt}
% Diagrams
\usepackage{graphicx}
\usepackage{float}
% Page headings
\usepackage{fancyhdr}
\pagestyle{fancy}
\renewcommand{\headrulewidth}{0pt}
\setlength{\headheight}{16pt}
%\newfontfamily\namefont[Scale=1.2]{Gloria Hallelujah}
\fancyhead{}

\usepackage{amsmath}

\graphicspath{ {images/} }

\usepackage{listings}
\lstdefinestyle{lablisting}{
  basicstyle=\scriptsize\ttfamily,
  numbers=left,
  stepnumber=1,
  otherkeywords={EOF, O\_RDONLY, STDIN\_FILENO, STDOUT\_FILENO, STDERR\_FILENO},
  numbersep=10pt,
  showspaces=false,
  showstringspaces=false
}

\newcommand{\specialcell}[2][l]{%
  \begin{tabular}[#1]{@{}l@{}}#2\end{tabular}}

\begin{document}

% Title page
\begin{titlepage}
\begin{center}

\textsc{Национальный исследовательский университет ИТМО\\[4mm]
Факультет программной инженерии и компьютерной техники}
\vfill
\textbf{Практическое задание №3\\[4mm]
по дисципение Теория Автоматов\\[4mm]
Канонический метод структурного синтеза\\[4mm]
}
\textit{Вариант 11\\[16mm]}
\begin{flushright}
Студент: Саржевский Иван
\\[2mm]Группа: P3302
\\[2mm]Преподаватель: Тропченко Александр Ювенальевич
\end{flushright}
\vfill
г. Санкт-Петербург\\[2mm]
2020 г.

\end{center}
\end{titlepage}

\section*{Цель}

Практическое освоение метода перехода от абстрактного автомата
к структурному автомату.

\section*{Задание}

Абстрактный автомат задан табличным способом. Причем абстрактный
автомат Мили представлен таблицами переходов и выходов, а абстрактный
автомат Мура - одной отмеченной таблицей переходов. Для синтеза
структурного автомата использовать функционально полную систему
логических элементов И, ИЛИ, НЕ и автомат Мура, обладающий полнотой
переходов и полнотой выходов. Синтезированный структурный автомат
представить в виде ПАМЯТИ и КОМБИНАЦИОННОЙ СХЕМЫ.

\section*{Исходные данные}

Согласно полученному варианту исходный автомат Мура задается
следующей таблицей переходов: 

\begin{center}
  \includegraphics{task.png}
\end{center}

\section*{Кодирование исходного автомата двоичными кодами}

\subsection*{Входной алфавит}

\begin{tabular}{| c | c |}
  \hline
  & $x_1$\\
  \hline
  $z_1$ & \texttt{0}\\
  $z_2$ & \texttt{1}\\
  \hline
\end{tabular}

\subsection*{Выходной алфавит}

\begin{tabular}{| c | c | c |}
  \hline
  & $y_1$ & $y_2$\\
  \hline
  $w_1$ & \texttt{0} & \texttt{0}\\
  $w_2$ & \texttt{0} & \texttt{1}\\
  $w_3$ & \texttt{1} & \texttt{0}\\
  $w_4$ & \texttt{1} & \texttt{1}\\
  \hline
\end{tabular}

\subsection*{Состояния}

\begin{tabular}{| c | c | c | c |}
  \hline
  & $Q_1$ & $Q_2$ & $Q_3$\\
  \hline
  $a_1$ & \texttt{0} & \texttt{0} & \texttt{0}\\
  $a_2$ & \texttt{0} & \texttt{0} & \texttt{1}\\
  $a_3$ & \texttt{0} & \texttt{1} & \texttt{0}\\
  $a_4$ & \texttt{0} & \texttt{1} & \texttt{1}\\
  $a_5$ & \texttt{1} & \texttt{0} & \texttt{0}\\
  $a_6$ & \texttt{1} & \texttt{0} & \texttt{1}\\
  $a_7$ & \texttt{1} & \texttt{1} & \texttt{0}\\
  $a_8$ & \texttt{1} & \texttt{1} & \texttt{1}\\
  \hline
\end{tabular}

\section*{Таблицы переходов и выходов соответствующего структурного автомата}

После кодирования исходного абстрактного автомата
Мура построим таблицы переходов и выходов структурного автомата.

\begin{tabular}{|*{9}{c|}}
  \hline
  $x_1/Q_1Q_2Q_3$ & 000 & 001 & 010 & 011 & 100 & 101 & 110 & 111\\\hline
  0 & 100 & 101 & 111 & 100 & 000 & 001 & 011 & 110\\\hline
  1 & 001 & 010 & 111 & 110 & 110 & 111 & 111 & 111\\\hline
\end{tabular}

\begin{tabular}{|*{9}{c|}}
  \hline
  $x_1/Q_1Q_2Q_3$ & 000 & 001 & 010 & 011 & 100 & 101 & 110 & 111\\\hline
  0 & 01 & 00 & 00 & 01 & 00 & 10 & 00 & 11\\\hline
  1 & 01 & 00 & 00 & 01 & 00 & 10 & 00 & 11\\\hline
   & $y_1y_2$ & $y_1y_2$ & $y_1y_2$ & $y_1y_2$ & $y_1y_2$ & $y_1y_2$ & $y_1y_2$ & $y_1y_2$\\\hline
\end{tabular}

\section*{Вывод}

\end{document}
